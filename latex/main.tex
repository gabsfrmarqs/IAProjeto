\documentclass[12pt]{article}

\usepackage{sbc-template}
\usepackage{graphicx,url}
\usepackage[utf8]{inputenc}
\usepackage[brazil]{babel}

     
\sloppy

\title{Preparação e Pré-processamento de Dados para Análise Preditiva entre Jogos Indie e Multi-A}

\author{Gabriel de França Marques (RA: 10395270)\inst{1}, Henrique Magno dos Santos(RA: 10335286)\inst{1}, \\ 
Pedro Machado Gomes Caixeta (RA: 10314309)\inst{1}}

\address{Ciência da Computação (CC) -- Faculdade de Ciência e Informação (FCI) -- \\
  Universidade Presbiteriana Mackenzie (UPM)
  \email{10395270@, 10335286@,
  10314309@mackenzista.com.br}
}


\begin{document} 

\maketitle

\begin{abstract}
  This project aims to perform a predictive analysis using as a main basis a dataset of 
  1500 games from the Steam sales platform. The main objective is to predict and find a 
  pattern related to revenue, reviews, other data, and the price of games in the store.
\end{abstract}
     
\begin{resumo} 
  Este projeto apresenta como objetivo realizar uma análise preditiva utilizando
  como base principal um dataset de 1500 jogos da plataforma de vendas Steam. O
  objetivo majoritário é predizer e buscar um padrão relacionado à receita, avaliações, 
  outros dados e o preço dos jogos na loja.
\end{resumo}


\section{Introdução}
\subsection{Contextualização} \label{sec:contextualizacao}

O mercado de jogos digitais tem se mostrado um campo fértil para a aplicação de técnicas 
de análise de dados e inteligência artificial. A capacidade de prever métricas-chave, 
como receita, avaliações e preço de jogos, pode trazer insights valiosos para 
desenvolvedores e publishers, auxiliando-os na tomada de decisões estratégicas.

Neste projeto, o objetivo é realizar uma análise preditiva utilizando um dataset de 1500 
jogos da plataforma Steam. A escolha desse dataset se justifica pela relevância e 
disponibilidade de informações detalhadas sobre o mercado de jogos digitais. O estudo e 
 previsão dessas métricas relevantes para o sucesso de um jogo podem contribuir 
 significativamente para o setor.

\subsection{Justificativa} \label{sec:firstpage}

O estudo e a previsão de métricas relevantes para o sucesso de um jogo, como receita, 
avaliações e preço, podem trazer insights valiosos para o setor.

\subsection{Objetivo}

O objetivo deste projeto é realizar uma análise preditiva utilizando um dataset de 
jogos da plataforma Steam, buscando encontrar padrões e prever informações 
importantes para o sucesso de um jogo.

\subsection{Opção do projeto}

A escolha deste dataset de jogos da Steam se justifica pela relevância e 
disponibilidade de informações detalhadas sobre o mercado de jogos digitais.

\section{Descrição do Problema}

O principal problema a ser abordado neste projeto é a identificação de fatores-chave 
que influenciam a receita, as avaliações e o preço dos jogos na plataforma Steam. 
Além disso, pretende-se desenvolver modelos preditivos capazes de estimar essas 
métricas com base nas características dos jogos.

\section{Dataset}

O dataset utilizado neste projeto contém informações de 1500 jogos da plataforma 
Steam. Serão realizadas etapas de limpeza, tratamento de dados ausentes, 
transformação de variáveis e análise exploratória para preparar o dataset para a 
modelagem preditiva.

\section{Metodologia e Resultados Esperados}

Serão aplicadas técnicas de aprendizado de máquina supervisionado, como regressão 
linear e árvores de decisão, para prever as métricas de receita, avaliações e 
preço dos jogos. Espera-se obter modelos preditivos com bom desempenho, capazes 
de fornecer insights valiosos para o mercado de jogos digitais.

\section{Bibliografia}

O dataset foi obtido em \cite{topcu2024top1500}.
Os slides de "Preparação e Pré-processamento dos dados" utilizados como base de 
estudo foram ministrados pelo Prof. Dr. Ivan Carlos Alcântara de Oliveira, sendo 
disponibilizados nas seguintes partes: 
\cite{professor_slides_2024parte1}, \cite{professor_slides_2024parte2} e 
\cite{professor_slides_2024parte3}.

\bibliographystyle{sbc}
\bibliography{main}

\end{document}
